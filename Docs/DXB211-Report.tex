\documentclass[12pt,a4paper]{article}
\usepackage[utf8]{inputenc}
\usepackage{graphicx}

\begin{document}
	\begin{titlepage}
		\begin{center}
			\includegraphics[width=0.5\textwidth]{QUT.jpg}\\
			[0.03\textheight]  
			\Large\textbf{Bachelor of IT (Computer Science)}\\
			\Large\textbf{Assignment 1 - Programming Sketchbook}\\
			\large\textbf{DXB211 - Creative Coding}\\
			[0.02\textheight]
			\large\textsl{Dane Madsen}\\
			\large\textsl{n10983864@qut.edu.au}
		\end{center}
		
	\end{titlepage}

	\section{Creative Process}
		\subsection{Active Drawing}
			The initial inspiration for this sketch came from halfbrick studios' Fruit Ninja.
			I also played around with a number of other concepts such as a snake game, an airport 
			manager game and a tower defence game. Ultimately I decided to go with a fruit ninja 
			game because it was the simplest to implement and A lot of the elements of the other 
			concepts were too far out of scope for this assignment.\\\\Initially with the knife 
			swipe mechanic I had a lot of trouble getting the swipe to work properly because the 
			method I was using to limit the line length was not working as intended and the swipe 
			would get cut off prematurely. I ended up changing the method to cull the lines after 
			25 draw cycles and this worked much better.I also encountered trouble working out 
			how to actually move the 'fruits' and how to detect collisions with the fruits. However, 
			I was able to solve both of these issues by researching online.
		
		\subsection{Recombination Effect}
			For this second sketch my idea was to make a sketch where initially the image is heavily 
			pixelated, but as the user draws over the image the pixels become smaller revealing the 
			original image. I chose this style for no other reason than because it reminded me of pixel
			art from indie games such as Stardew Valley.\\\\Though this sketch is quite different to the 
			first sketch, they both have very similar code which made development of this sketch easier.
			However, I still encountered a number of problems when developing this sketch. The most 
			pressing issue was the culling of the pixels. I initially tried to cull pixels the same 
			way fruits are culled in the Active Drawing sketch but for some reason this method resulted 
			in a few pixels failing to subdivide so I switched to a method that culls the pixels from 
			within the subdivide function.

		\subsection{Moveable Types}
			For this last sketch I opted for a much simpler concept than the previous two sketches.
			I decided to make a sketch that functioned as a simple digital clock with the edition 
			that every second the decorative circles around the clock would change colour.\\\\This 
			sketch was the easiest to develop out of the three sketches but i believe it still
			adequatly demonstrates my understanding of the concepts of moveable types.
	
	\section{Relation to Field}
		The sketches I have created relate to the field of creative practices by showcasing a variety
		of aspects of creativity. The first sketch (Active Drawing) showcases games design by demonstrating
		the potential of using active drawing as a simple game mechanic. The second sketch (Recombination 
		Effects) showcases image manipulation by demonstrating the potential of using sampling and active 
		drawing to create a new image. Finally, the third sketch (Moveable Types) showcases design by 
		demonstrating the potential of using time as a design element.
	
	\section{References}
	All images used in this assignment (with exception to the QUT logo) were generated using stable diffusion 
	and thus have no copyright.
\end{document}
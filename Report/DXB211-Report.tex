\documentclass[12pt,a4paper]{article}
\usepackage[utf8]{inputenc}
\usepackage{graphicx}

\begin{document}
	\begin{titlepage}
		\begin{center}
			\includegraphics[width=0.5\textwidth]{QUT.jpg}\\
			[0.03\textheight]  
			\Large\textbf{Bachelor of IT (Computer Science)}\\
			\Large\textbf{Assignment 1 - Programming Sketchbook}\\
			\large\textbf{DXB211 - Creative Coding}\\
			[0.02\textheight]
			\large\textsl{Dane Madsen}\\
			\large\textsl{n10983864@qut.edu.au}
		\end{center}
		
	\end{titlepage}

	\section{Creative Process}
		\subsection{Active Drawing}
			The initial inspiration for this sketch came from halfbrick studios' Fruit Ninja.
			I also played around with a number of other concepts such as a snake game, an airport 
			manager game and a tower defense game. Ultimately I decided to go with a fruit ninja 
			game because it was the simplest to implement and A lot of the elements of the other 
			concepts were too far out of scope for this assignment.\\\\Initially with the knife 
			swipe mechanic I had a lot of trouble getting the swipe to work properly because the 
			method I was using to limit the line length was not working as intended and the swipe 
			would get cut off prematurely. I ended up changing the method to use cull lines after 
			25 draw cycles and this worked much better.\\\\I also encountered trouble working out 
			how to actually move the 'fruits' and how to detect collisions with the fruits. However, 
			I was able to solve both of these issues by researching online.
		
		\subsection{Recombination Effect}
			For this second sketch I initially had the idea that I would have a pixilated image which 
			I could draw over to reduce the pixilation 

		\subsection{Moveable Types}
	
	\section{Statement of Completeness}
		Assignment submission is complete. Application fulfils all required functionality. 
		The fire alarm system attempts to follows NASA The power of 10, ISO 26262-6:2018
		and MISRA C to a standard practical for the assessment.
	
	\section{Fire Alarm Safety Assessment}
		\subsection{Safety-critical standards provided fire alarm fails}
			\subsubsection{NASA the power of 10}
				\begin{enumerate}
					\item\textbf{Avoid complex flow constructs, such as goto and recursion.}\\Fail – firealarm.c uses goto on line 159 and deletenodes() is a recursive function.
					\item\textbf{All loops must have fixed bounds. This prevents runaway code.}\\Fail – firealarm.c uses unbound loops on lines 61, 134, 157 and 189.
					\item\textbf{Avoid heap memory allocation.}\\Fail – firealarm.c uses malloc on lines 67, 82, 91, 152 and 176.
					\item\textbf{Restrict functions to a single printed page.}\\Fail – functions tempmonitor() on line 56, and main() on line 147 go well over one page.
					\item\textbf{}
				\end{enumerate}

\end{document}